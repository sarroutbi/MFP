\documentclass[10pt]{article}

% Puedes entregar la plantilla en inglés, si así lo deseas.
\usepackage[english]{babel}
\usepackage[utf8]{inputenx}

\begin{document}

\title{MSWL Final Project: Contribution to the OpenDaylight Community}
\author{Sergio Arroutbi Braojos}
\date{\today}


\maketitle

\begin{abstract}

This Final Project is entitled to show the process that a reader goes through for turning into a contributor and/or collaborator.\\
\\
By means of studying author's capabilities and motivations, it will be shown why the community selected is important for the author, and why studying it is important.\\
\\
Apart from that, it will also be shown how contributions can be started being accomplished, and to which level contributions can be performed in the lapse of time that this Final Project is under development.

\end{abstract}

\section{Objective}

Main objective for this Final Project is:

\begin{center}
\bf{Achieve contributions to the OpenDayLight opensource project}
\end{center}

Este objetivo se puede dividir en los siguientes subobjetivos:

\begin{enumerate}
  \item Subobjective 1. Deep study of the OpenDayLight community. Community, technical architecture, channels, interaction, etc.
  \item Subobjective 2. Study of abilities and motivations, and idenfication of the "best fit" contributions that can be performed.
  \item Subobjective 3. Achieve high quality contributions to the OpenDayLight project.
  \item Subobjective 4. Perform a "look back" of the contributions performed, and identify the entry barriers for the project.
\end{enumerate}


\section{Motivation}

The motivations for performing this Final Project from the author's view are basically two:\\
\begin{enumerate}
\item{Start  contributing to an opensource community}. As a way of restitution to the community of all the benefits that the author consider having received.
\item{Obtain acknowledge on SDN \& NVF technology}. As they are considered key technologies for the coming years.
\end{enumerate} 
Start contributing to OpenDayLight community is, somehow, a way to accomplish both motivations, as it is an Open Source project, hosted by the Linux Foundation, and the project is related to the SDN \& NVF technologies described.

\section{Experience on this environment}

In terms of knowledge experience, no strong capabilities are hold by the author, even considering that this kind of technology (SDN \& NVF) are incipient technologies, and the OpenDayLight project is, moreover this, a recently created community.\\
\\
However, author can contribute to the community, taking into account that the author:
\begin{itemize}
\item{Has strong experience on Networking.}
\item{Has strong experience on Computer Programming concepts.}
\item{Has experience on Python programming.}
\item{Has experience on using Git and Gerrit.}
\end{itemize}

\section{Temptative planification}

Incluye una planificacitón temporal tentativa. 

\begin{enumerate}
  \item Initiate Memorandum document and time planning : November 2013
  \item Analysis of the OpenDayLight community : December 2013 - January 2014
  \item Study of capabilities and motivations, and identification of areas to contribute: Januaray 2014
  \item Contributions Accomplishment : February 2014 - May 2014
  \item Contribution achievment study and Final Project Conclussions : June 2014
  \item Memorandum Delivery: June 2014
\end{enumerate}

\section{Other issues}

No other issues have been considered to accomplish the correct execution of this Final Project.

\end{document}
