\documentclass[a4paper, 12pt]{book}
\usepackage[a4paper, left=2.5cm, right=2.5cm, top=3cm, bottom=3cm]{geometry}
%\usepackage[a4paper]{geometry}
\usepackage{times}
\usepackage{color}
\usepackage[usenames,dvipsnames,svgnames,table]{xcolor}
\usepackage[utf8]{inputenc}
\usepackage[textwidth=2cm]{todonotes}
\usepackage[hyphens]{url}
\usepackage[english]{babel}
%\usepackage[dvipdfm]{graphicx}
\usepackage{float}  
\usepackage[nottoc, notlot, notlof, notindex]{tocbibind}
\usepackage{latexsym}  %% Logo LaTeX
\usepackage{graphicx}
\usepackage{multirow}
\usepackage[colorlinks,bookmarksopen]{hyperref}
%\usepackage[svgnames]{xcolor}

%% PDF metadata
\hypersetup{
  pdftitle={The OpenDaylight Open Source Project},
  pdfauthor={Sergio Arroutbi Braojos},
  pdfcreator={Master on Libre Software (URJC), Universidad Rey Juan Carlos},
  pdfproducer=PDFLaTeX,
  pdfsubject={Libre Software},
  %%% change colors to darker ones (for printing in B/W)
  linkcolor=Sepia,
  citecolor=OliveGreen,
  filecolor=violet,
  urlcolor=blue
}
%%

% Alter some LaTeX defaults for better treatment of figures:
    % See p.105 of "TeX Unbound" for suggested values.
    % See pp. 199-200 of Lamport's "LaTeX" book for details.
    %   General parameters, for ALL pages:
    \renewcommand{\topfraction}{0.9}	% max fraction of floats at top
    \renewcommand{\bottomfraction}{0.8}	% max fraction of floats at bottom
    %   Parameters for TEXT pages (not float pages):
    \setcounter{topnumber}{2}
    \setcounter{bottomnumber}{2}
    \setcounter{totalnumber}{4}     % 2 may work better
    \setcounter{dbltopnumber}{2}    % for 2-column pages
    \renewcommand{\dbltopfraction}{0.9}	% fit big float above 2-col. text
    \renewcommand{\textfraction}{0.07}	% allow minimal text w. figs
    %   Parameters for FLOAT pages (not text pages):
    \renewcommand{\floatpagefraction}{0.7}	% require fuller float pages
	% N.B.: floatpagefraction MUST be less than topfraction !!
    \renewcommand{\dblfloatpagefraction}{0.7}	% require fuller float pages

\frenchspacing

\title{The OpenDaylight Open Source Project}
\author{Sergio Arroutbi Braojos}

\renewcommand{\baselinestretch}{1.5}  

\begin{document}

%\renewcommand{\refname}{Bibliography}  
\renewcommand{\appendixname}{Appendix}

%%%%%%%%%%%%% COVER %%%%%%%%%%%%%%%%
\begin{titlepage}
\begin{center}
\begin{tabular}[c]{c c}
%\includegraphics[bb=0 0 194 352, scale=0.25]{logo} &
\includegraphics[scale=0.25]{img/logo.png} &
\begin{tabular}[b]{l}
\Huge
\textsf{UNIVERSIDAD} \\
\Huge
\textsf{REY JUAN CARLOS} \\
\end{tabular}
\\
\end{tabular}

\vspace{3cm}

\Large
Máster Universitario en Software Libre

\vspace{0.4cm}

\large
Curso Académico 2013/2014

\vspace{0.8cm}

Proyecto Fin de Máster

\vspace{2.5cm}

\LARGE
The OpenDaylight Open Source Project

\vspace{4cm}

\large
Autor: Sergio Arroutbi Braojos \\
Tutor: Dr. Gregorio Robles
\end{center}
\end{titlepage}
%%%%%%%%%%%%%%%%%%%%%%%%%%%%%%%%%%%%%%
\newpage
~

\newpage
~
\thispagestyle{empty}
\vspace{3cm}
\begin{flushright}
\textbf{\textit{Agradecimientos}} \\
\textit{Al equipo de Libresoft en la Universidad Rey Juan Carlos, 
por su afán en enseñar el qué y el porqué del Software Libre.\\
A mi familia y a mi pareja, por su apoyo incondicional.}
\vspace{2cm}

\textbf{\textit{Dedicatoria}} \\
\textit{Para todos aquellos que hacen posible el fenómeno del Software Libre}
\end{flushright}

\newpage
~


\newpage
~
\thispagestyle{empty}
\vspace{12cm}
\begin{flushright}

(C) 2014 Sergio Arroutbi Braojos. Some rights reserved.

This document is distributed under the Creative Commons
Attribution-ShareAlike 3.0 license,
available in \url{http://creativecommons.org/licenses/by-sa/3.0/}

Source files for this document are available at
\url{http://github.com/MFP/opendaylight.tex}
\end{flushright}

\newpage
~

\tableofcontents  

\listoffigures  

\listoftables 

%%%%%%%%%%%%%%%%%%%%%%%%%%%%%%%%%%%%%%

\chapter*{Summary}
\markboth{SUMMARY}{SUMMARY}
\label{chap:summary}
The main goal of this work is to perform a deep analysis about the OpenDaylight Open Source Project. This collaborative project, hosted by the Linx Foundation, has been created in order to achieve one mission: \textbf{Develop an Open Source Programmable Networking Platform}.

In this document, a detailed study of the different Open Source aspects having to do with project of this type of characteristics will be analyzed. Examples of this aspects, are, for instance, the licensing mechanism adopted by the project, the community behind the project, descriptive statistics about the project, the economic aspects around the project, as well as the technical state of the proyect and how OpenDaylight has progressed from its foundation date.


\chapter*{Resumen}
\markboth{RESUMEN}{RESUMEN}
\label{chap:resumen}

El principal objetivo de este trabajo es realizar un análisis detallado del projecto de código abierto OpenDaylight. Este proyecto colaborativo, perteneciente a la Linux Foundation, ha sido creado para una misión principal: \textbf{Desarrollar una plataforma de Código Abierto para Redes Programables}.

En este documento, un estudio pormenorizado de los distintos aspectos asociados al Código Abierto asociados para un proyecto de estas características serán analizados. Un ejemplo de los aspectos a estudiar será el mecanismo de sistema de licencias, la comunidad que reside detrás del proyecto, estadísticas descriptivas del proyecto, los aspectos económicos, si los hubiera, alrededor del proyecto, así como el estado a nivel técnico o cómo ha evolucionado OpenDaylight desde la fecha de su fundación.

%%%%%%%%%%%%%%%%%%%%%%%%%%%%%%%%%%%%%%

\chapter{Introduction}
\label{chap:introduction}

\section{Terminology}
\label{sec:terminology}

\subsection{Open Source Programmable Networking}
\label{subsec:freesoftware}
In the same way Cloud Computing means a revolution in Computer Science, where computing resources are considered as flexible facilities to provide different kind of services, Networking is evolving in the same way. Networks hardware is considered to be a resource that is flexible and easily programmable in order to adapt to the specific Networking necessities. \textbf{SDN} and \textbf{NFV} ~\cite{SDN}\dots technologies have been strongly brougth to foreground in Networking Science, in order to provide,on the one hand, management of the network services through abstraction of lower level functionality and characteristics, and, on the other hand to provide a network architecture virtualization technologies to simulate nodes exisiting on a network.

Around this technologies, The OpenDaylight Open Source Project, hosted by the Linux Foundation, has appeared to provide mechanisms not only to use previous described technology, but also to guarantee all the potential users of this technology the Freedoms that Open Source means, i.e.:
\begin{itemize}
 \item Freedom to use the program, for any purpose
 \item Freedom to study and adapt the programs (modify)
 \item Freedom to distribute the program to others
 \item Freedom to distribute to others the modified versions of the program
\end{itemize}
Analyzing The OpenDaylight Open Source Project is a good oportunity to investigate, on the one hand, an incipient technology and how an also incipient Open Source Project can influence not only on that technology, but also on the different aspects around a technology, as people involved on the technology develeopment (Community), the impact on the Economic aspects around the technology, and, above all, the aspects that Open Source itself supposes for this kind of projects.

In figure~\ref{fig:ScenarioLocalization} we can find a diagram summarizing this
two different scenarios.

  \begin{center}
   \begin{figure}[htbp]
   \begin{center}
     \includegraphics[width=15cm]{img/ScenarioLocalization.png}
     \caption{You can create nice diagrams using FreeMind}     
\label{fig:ScenarioLocalization}
   \end{center}
    \end{figure}
   \end{center}

\missingfigure{If you want to include a figure but still didn't create it, use
missingfigure}

\section{About this document}
\label{sec:about}

\subsection{Document structure}

In order to provide a detailed analysis of The OpenDaylight Open Source Project, this work contains different chapters to describe the different important aspects aroud the project from an Open Source perspective:

\begin{table}[htbp]
\footnotesize
\begin{center}
\begin{tabular}{|l|l|l|}
\hline
\textbf{Chapter Name} & \textbf{Description} \\ \hline
Introduction & A complete introductory overview of The OpenDaylight Open Source Project \\ \hline
OpenDayligth Technical Aspects & Study of the technology behind the project and its scope \\ \hline
OpenDaylight Legal Aspects & Complete analysis of the license or licenses used in OpenDaylight \\
& and the different advantages and disadvantages this kind of licensing \\ \hline
OpenDayligth Economic Aspects & Detailed study of economic aspects around the project \\ \hline
OpenDayligth Community Management & Analysis of the community, its organization, communication and politics  \\ \hline
OpenDaylight Project Evaluation & Statistics around the OpenDayligth project, such as number of commiters, \\
& number of open issues, mail lists  \\ \hline
\end{tabular}
\end{center}
\caption{Document Structure}
\label{tab:i18nformats}
\end{table}

\subsection{Scope}
\label{subsec:scope}
This document is not entitled to perform a complete description of the different protocols and technologies that OpenDaylight uses in order to achieve the programmability network platform it pretends to provide. They will be smoothly analyzed in order to clarify how OpenDaylight works, but not all of the technologies will be described, and those described will not be done in deep.

Beyond the purely technical aspects, this work pretends to focus on analyzing OpenDaylight project from an Open Source perspective, analysing the pros and cons of Open Source, and the different aspects that an Open Source Project faces.

\subsection{Methodology}
\label{subsec:methodology}
Different tools and documentation have been used in order to perform this work. Docmentation used have been basically the different Web Pages available around The OpenDaylight Open Source Project~\cite{OpenDaylight}. A complete description of the documentation used will be provided in the Bibliography.

Regarding tools, apart from Web-Browsers used to navigate through the project documentation, the different metrics obtained around OpenDaylight project have been obtained through MetricsGrimoire~\cite{MetricsGrimoire}. In particular, among the tools existing on MetricsGrimoire, CVSanaly~\cite{CVSanaly}, Bicho~\cite{Bicho} and MailingListStats~\cite{MailStats} were used.

%%%%%%%%%%%%%%%%%%%%%%%%%%%%%%%%%%%%%%
\chapter{Goals and Objectives}
\label{chap:Goals} 
\section{General Objectives}

The general objectives of this work are, basically, on the one hand, acquiring knowledge, competence and skills around OpenDaylight, while, on the other hand, analysing the project from an Open Source perspective.

\section{Subobjectives}
%%%%%%%%%%%%%%%%%%%%%%%%%%%%%%%%%%%%%%
In order to achieve the objectives this work pursues next operative objectives have been identified:

\begin{itemize}
\item{Acquire competence on OpenDaylight project}
\item{Analyze OpenDayligth project from an Open Source perspective}
\item{Extract the most significant statistics in OpenDaylight project to determine its state of the art}
\end{itemize}

\subsection{Acquire competence on OpenDaylight project}
\begin{itemize}
 \item Perform an overall description of the OpenDaylight project.
 \item Acquire competence on OpenDaylight documentation, and sinthesize the most important aspects of the project.
 \item Analyze the technical aspects of the project and determine its Ease of Use.
\end{itemize}

\subsection{Analyze OpenDaylight project from an Open Source perspective}
\begin{itemize}
 \item Analyze the licensing model followed by the project
 \item Study the economic aspects behind this project
 \item Determine the different aspects behind the project's community
\end{itemize}

\subsection{Statistics and measures of the OpenDaylight project}
\begin{itemize}
 \item Perform a complete measures compilation of the project
 \item Evaluate the State of the Art of the project based on its measures
\end{itemize}

\chapter{Getting to the point}
\label{chap:point}

\section{Structuring the document in sections}

Example for URL: more information here: 
\url{http://developer.android.com/guide/topics/resources/localization.html}.

Example of table: In table~\ref{tab:i18nformats} you can find a list
of formats for localization~\cite{GPL}.

Example of \textbf{footnote}\footnote{\url{
http://translate.sourceforge.net/wiki/pootle/}, this is a footnote}, 
in ~\LaTeX~ suits nice.

Example of an in-document reference:
in section~\ref{chap:introduction} introductory concepts are explained.

%%%%%%%%%%%%%%%%%%%%%%%%%%%%%%%%%%%%%%

\chapter{Conclusions}
\label{chap:conclusions}

%%%%%%%%%%%%%%%%
% Review goals and objectives

\section{Evaluation}
%%%%%%%%%%%%%%%%%%%%%%%%%%%%%%%%%%%%%%

Review goals and objectives

%%%%%%%%%%%

\section{Lessons learned}
\label{sec:lessons}

\subsection{Lesson 1}
\begin{itemize}
 \item Explain here what anybody may learn from this document
\end{itemize}

\subsection{What I learned}
\begin{itemize}
 \item Explain what you learned writing this document
\end{itemize}

\subsection{Knowledge and skills acquired in the M.Sc. studies that helped me on
this work}
\begin{itemize}
 \item You can go subject by subject, explained in which aspects helped
you to do/write this project
 \item Or just mention only the main aspects/subjects

\end{itemize}

\section{Future work}
\label{sec:future}

\subsection{More on...}

If there is any aspect that is not fully covered, explain how it could be
better covered.

\subsection{Other aspects}

See the section about ``Scope'' and explain here other aspects that you didn't
cover but it could be nice to work on.


\subsection{Other focuses}

For example study similar projects, or apply other tools to do the same work.


%%%%%%%%%%%%%%%%%%%%%%%%%%%%%%%%%%%%%
\appendix

%%%%%%%%%%%%%%%%%%%%%%%%%%%%%%%%%%%%%%
\chapter{First Appendix}

\chapter{Second Appendix}

%%%%%%%%%%%%%%%%
% BIBLIOGRAPHY %
%%%%%%%%%%%%%%%%

\bibliographystyle{alpha}
\bibliography{bibliography}
\label{Bibliography}
\end{document}
